\section{\pmv}
\label{pMarineViewer}
For programming reference, the collaboration diagram for \pmv\ is included in Fig.~\ref{fig:pmvdiagram}.

\img[width=\linewidth]{figures/pmvdiagram}{A collaboration diagram for \pmv}{fig:pmvdiagram}

\subsection{Configuration}
\pmv\ is launched by calling {\tt \pmv\ file.moos} (contrary to the usage instructions, no other arguments are read from the command line).  Also, the file must be named .moos, or it will not be loaded by \pmv.

\subsubsection{MOOS Configuration Block}
The \pmv\ MOOS configuration block looks like:
\scriptsize
\begin{verbatim}
//------------------------------------------
// pMarineViewer config block
{ 
  AppTick   = 4
  CommsTick = 4

  TIF_FILE  = Default.tif  
  VEHICOLOR = nyak200, darkblue
  VEHICOLOR = nyak201, hex, 08, a4, ff
  VEHICOLOR = nyak204, .450, .132, .55
}
\end{verbatim}
\normalsize
\begin{hangpar}{\pin}{\var{TIF\_FILE: }}
Optional.  Default value is ``Default.tif''.  The path to the image file to be used as the background for the display window.  
\end{hangpar}
\begin{hangpar}{\pin}{\var{VEHICOLOR: }}
Optional.  Sets the color of a labelled object.  The format of this string is ``label, \{colorname OR hex[:], ff, ff, ff OR .050, .071, .125\}'' See Appendix~\ref{app:colormap} for a list of color names.
\end{hangpar}

\subsubsection{Background Image Data}
Each tif file should have a .info file with the same name.  Comment lines are prefaced with ``//''.  This text file can have the following entries:
\begin{hangpar}{\pin}{\var{img\_centx and img\_centy: }}
The pixel that represents the origin of the image.  UNITS UNKNOWN.
\end{hangpar}
\begin{hangpar}{\pin}{\var{img\_offset\_x and img\_offset\_y: }}
FEATURE UNDOCUMENTED.
\end{hangpar}
\begin{hangpar}{\pin}{\var{centlat and centlon: }}
The latitude and longitude (in +N and +E degrees) of the center of the image.
\end{hangpar}

For example:
\scriptsize
\begin{verbatim}
img_centx   = 0.495850  
img_centy   = 0.509000
img_meters  = 0.048828
//img_meters  = 0.48828
img_centlat = 42.35849
img_centlon = -71.08759333
\end{verbatim}
\normalsize

\subsection{Menus and Interface}
All \pmv\ keyboard shortcuts are documented in the menu system.  The mouse performs no action in \pmv.

\subsection{MOOS Variables}
\subsubsection{Subscribes}
In all of the requests to plot a figure, if a label is given, requesting to plot that label again with different values will cause the figure to move to that position instead of drawing a duplicate.

\begin{hangpar}{\pin}{\var{AIS\_REPORT and AIS\_REPORT\_LOCAL: }}
An AIS report is a string of the form ``NAME=name, TYPE=type, X=valx, Y=valy, SPD=speed, HDG=heading, DEPTH=depth''.  The name value must match the sending community's name.  All of the variables except type are required for a valid AIS report.  For type, \pmv\ currently knows how to draw types ``kayak'' and ``auv''.  All other types are drawn with a default symbol.
\end{hangpar}
\begin{hangpar}{\pin}{\var{GRID\_CONFIG: }}
Configures \pmv\ to plot a new XYGrid on the display.  The string must be a valid XYGrid configuration string (of the form ``polygon\_string@unit\_string[@initial\_value]'')  A polygon\_string is ``poly[gon]: [label,LABELNAME:] segment\_list''.  A segment\_list is a colon seperated list of comma separated x,y pairs (e.g. 4,5.5:1,2.2).  The unit\_string is the dimensions of the rectangle to place inside the bounding polygon.  It is of the form ``x\_width, y\_width''.  Usually, these are the same value.  The label is used to uniquely identify the grid.
\end{hangpar}
\begin{hangpar}{\pin}{\var{GRID\_DELTA: }}
A string to update the grid.  It is of the form ``LABELNAME@index, old\_val, new\_val[,old\_utility, new\_utility][:index, old\_val, new\_val...]''.
\end{hangpar}
\begin{hangpar}{\pin}{\var{VIEW\_POLYGON: }}
Plots the specified polygon.  The string is a valid XYPolygon initialization string.  Polygons can also be of the form ``radial:xval, yval, radius, num\_points[,snap\_value[,LABELNAME]]'' to approximate a circle.  (An arc can also be plotted, see XYPolygon.cpp for details.)
\end{hangpar}
\begin{hangpar}{\pin}{\var{VIEW\_SEGLIST: }}
Plots the specified seglist.  The string is a valid XYSeglist initialization string.  It is of the form ``[label,LABELNAME:]segment\_list''.  It can also be of the zigzag form (not described here.  See XYSegList.cpp for a description.
\end{hangpar}
\begin{hangpar}{\pin}{\var{VIEW\_POINT and VIEW\_CIRCLE: }}
VIEW\_CIRCLE IS NOT IN SUBSCRIPTION LIST.  VIEW\_POINT plots a dot.  The string is a valid XYCircle initialization string.  It is of the form ``x\_val,y\_val,radius[,LABELNAME]''.
\end{hangpar}
\begin{hangpar}{\pin}{\var{TRAIL\_RESET: }}
Forces \pmv\ to ``forget'' the current trails for any vehicles.
\end{hangpar}
\subsubsection{Publishes}
No MOOSDB writes are created by \pmv.
