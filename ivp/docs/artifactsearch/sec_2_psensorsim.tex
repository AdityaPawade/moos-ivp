\section{pSensorSim}
\label{pSensorSim}

pSensorSim is composed of two C++ classes, ArtifactField and SensorModel.  Most users will not directly use these classes and will instead interact with them through the MOOSApp pSensorSim.

\subsection{ArtifactField}
\label{classArtifactField}
ArtifactField simulates an artifact field.  Internally, it is a vector of strings, where each string represents one artifact.  An artifact string consists of a comma separated list of equal-sign delimited variable-value pairs.  For example, ``Var1=val1,Var2=val2,Var3=val3''.  This structure makes it easy to add new traits to artifacts without having to change much code in other segments.

ArtifactField can return a list of artifacts within a 2D rectangle or circle when the artifact strings contain both ``X=xval'' and ``Y=yval'' (e.g ``X=10,Y=4.5,TYPE=magnetic'').

\subsubsection*{Public Member Functions}
\begin{itemize}
\item void {\bf add\-Artifact} (std::string)
\begin{list}\small\item\em Puts an artifact into the field. \item\end{list}
\item void {\bf add\-Artifact} (double, double)
\begin{list}\small\item\em Constructs the proper string from an {\em x\/}, {\em y\/} pair. \item\end{list}
\item std::string {\bf get\-Artifact} (int) const 
\begin{list}\small\item\em Returns the artifact at index {\em i\/}. \item\end{list}
\item int {\bf size} () const
\begin{list}\small\item\em Returns the number of artifacts in the field. \item\end{list}
\item std::vector$<$ std::string $>$ {\bf get\-Artifactbox} (double, double, double, double) const
\begin{list}\small\item\em Returns a vector of all artifacts within the 2D, X,Y box specified by the parameters. \item\end{list}
\item std::vector$<$ std::string $>$ {\bf get\-Artifactcircle} (double, double, double) const
\begin{list}\small\item\em Returns a vector of all artifacts within the 2D, X,Y circle specified by the parameters. \item\end{list}\end{itemize}

\subsection{SensorModel}
\label{classSensorModel}
SensorModel models the output of a specified sensor on a given ArtifactField.  After creating a SensorModel object, the programmer initializes the sensor by calling {\bf set\-Sensor\-Model} with the name of the model to simulate (currently, only a fixed radius, guaranteed-detection sensor is modeled, ``fixedradius'') and setting the detection radius using {\bf set\-Sensor\-Radius}.  The programmer can query the sensor by calling {\bf query\-Sensor} with a query string that is determined by the sensor.  For the fixed-radius sensor, the query string should contain the current X and Y values, e.g. ``X=4.5,Y=1.3''.

\subsubsection*{Public Member Functions}
\begin{itemize}
\item bool {\bf set\-Sensor\-Model} (std::string const)
\begin{list}\small\item\em Currently accepted value is ``fixedradius''. \item\end{list}
\item void {\bf set\-Sensor\-Radius} (double)
\begin{list}\small\item\em The sensor radius must be a non-negative value. \item\end{list}
\item double {\bf get\-Sensor\-Radius} () const
\item std::vector$<$ std::string $>$ {\bf query\-Sensor} (std::string const, {\bf Artifact\-Field} const \&) const
\begin{list}\small\item\em \begin{description}
\item[Parameters:]
\begin{description}
\item[{\em Art\-Field}]is a reference to an artifact field \end{description}
\end{description}
\item\end{list}\end{itemize}

\subsection*{Private Member Functions}
\begin{itemize}
\item std::vector$<$ std::string $>$ {\bf query\-FRSensor} (std::string const, {\bf Artifact\-Field} const \&) const
\begin{list}\small\item\em A private method for querying a fixed-radius sensor. \item\end{list}\end{itemize}
\subsection*{Private Attributes}
\begin{itemize}
\item double {\bf d\-Sensor\-Radius}
\begin{list}\small\item\em The maxiumum effective sensor radius. \item\end{list}
\item std::string {\bf s\-Sensor\-Type}
\begin{list}\small\item\em A string holding the current sensor type. \item\end{list}\end{itemize}

\subsection{pSensorSim}
Combining these two classes and creating a MOOSApp, we get Fig.~\ref{fig:sensorsim}.

\img[width=\linewidth]{figures/sensorsim}{A class diagram for pSensorSim}{fig:sensorsim}

The pAntler configuration block for pSensorSim looks like this:
\begin{verbatim}
//------------------------------------------
// pSensorSim config block
ProcessConfig = pSensorSim
{
   AppTick   = 4
   CommsTick = 4
   
   ArtifactFile = mines.art
   Artifact = X=10,Y=10
   Sensor = FixedRadius
   Sensor_Radius = 10   
}
\end{verbatim}

\begin{itemize}
\item {\bf ArtifactFile} This is the path to a file that contains lines of the form ``Artifact = artifactstring''.  Blank lines are ignored.  Useful for adding randomly generated fields using artfieldgenerator.  See Appendix~\ref{app:artfieldgenerator}
\item {\bf Artifact} An artifact string to add to the artifact field.
\item {\bf Sensor} A string containing the sensor type to simulate.  Only FixedRadius is currently implemented.
\item {\bf Sensor\_Radius} The effective radius of the FixedRadius sensor.  Defaults to 10m, must be non-negative.
\end{itemize}
