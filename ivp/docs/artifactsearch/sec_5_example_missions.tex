\section{Example Missions}
\label{examples}

\subsection{Tutorial}
\label{ex:tutorial}
This example gives a tutorial on how one might go about creating and executing a mission to search for artifacts.

\subsubsection{Setup}
\label{ex:tutorial:setup}
The first step is to define the search area.  Using polyview, click on a few points (maintaining a convex polygon) to create the search area and export the polygon string.

\img[width=\linewidth]{figures/01polyview}{Defining the search area in polyview}{fig:01polyview}

We now generate a random artifact field for searching.  In the directory you want to run your mission file from:

\scriptsize
{\tt artfieldgenerator label,A:-119,-60:109,40:130,-97:-55,-156 .25 25 > mines.art}
\normalsize

This generates some random artifacts:

\scriptsize
\begin{verbatim}
head -4 mines.art 
ARTIFACT = X=92.5,Y=26.25
ARTIFACT = X=59.25,Y=-47.25
ARTIFACT = X=-30.25,Y=-60.25
ARTIFACT = X=88.5,Y=-85.25
\end{verbatim}
\normalsize

The next setup task is to create the MOOS mission file.  See Appendix~\ref{app:tutorialmission} for a working example.  The relevant portions are printed below:
\scriptsize
\begin{verbatim}
//------------------------------------------
// pSensorSim config block
ProcessConfig = pSensorSim
{
   AppTick   = 4
   CommsTick = 4
   
   ArtifactFile = mines.art
   Artifact = X=10,Y=10
   Sensor = FixedRadius
   Sensor_Radius = 10   
}

//------------------------------------------
// pArtifactMapper config block
ProcessConfig = pArtifactMapper
{
   AppTick   = 4
   CommsTick = 4
   
   GridPoly =  label,A:-119,-60:109,40:130,-97:-55,-156
   GridSize = 5.0
   GridInit = .5
}
\end{verbatim}
\normalsize

We also need to configure the helm to run a simple behavior to search over the search area.  To do this we first need to create a sequence of points for the vehicle to search over.  Load polyview with the previously defined search area (pass polyview a file with a line that reads ``Polygon = polystring'' to get it to load) and create a new ``polygon'' (not really a polygon as it will not be convex) whose points are the points in the lawn-mower pattern.  See Fig.~\ref{fig:02lawnmower} for an example.  Export this sequence of points and put it in the bhv file.  See Appendix~\ref{app:tutorialbehavior} for an example behavior.

\img[width=\linewidth]{figures/02lawnmower}{Defining the lawn-mower pattern in polyview.}{fig:02lawnmower}

\subsubsection{Launch}
\label{ex:tutorial:launch}
After getting setup for the mission, we invoke it with {\tt pAntler~mission.moos}.  The display should look like Fig.~\ref{fig:03missionstart}.  The blue grid is the search grid, the light blue dots are artifacts, and the circle around the kayak is the detection radius.  To start the helm, in iRemote, set {\tt Deploy = TRUE} (key 4), and then relinquish manual control (`o').

\img[width=\linewidth]{figures/03missionstart}{pMarineViewer at the beginning of an artifact search mission.  The blue grid is the search grid.  The bright-blue dots are artifacts.  The circle around the kayak is the 10m detection radius.}{fig:03missionstart}

The kayak will now loop through the points defined in the lawn-mower pattern and the various displays will update accordingly.  See Fig.~\ref{fig:04missionmiddle}

\img[width=\linewidth]{figures/04missionmiddle}{The various search components working together in the middle of a mission.}{fig:04missionmiddle}
