\documentclass[letterpaper,10pt]{article}
\usepackage{graphicx}
\usepackage{verbatim}

\title{Producing Debian packages for MOOS}
\author{Christian Convey (christian.convey@navy.mil)}
\date{2007-08-01}

\begin{document}

\maketitle

\tableofcontents

\parskip 7.2pt           % sets spacing between paragraphs

\section{Overview}
Debian, and related Linux distributions such as Ubuntu, use the APT package 
management system to let users easily download and install software.  
We now produce Debian packages for MOOS and (eventually) IvP to make it easier
for Debian / Ubuntu / etc. users to try out and work with MOOS/IvP software.

This document is both a HOWTO guide for producing MOOS's Debian package, and
a design document explaining how MOOS's build system has been modified to support
the production of Debian packages.

\section{The Packages}

\begin{center}
% use packages: array
\begin{tabular}{|l|l|}
\hline
Package name & Provides \\ 
\hline
\hline
moos-core & MOOSDB \\ 
moos-core-dev & MOOSLib, MOOSGenLib, header files \\ 
moos-essentials & pMOOSBridge, pAntler, pLogger, pScheduler \\ 
moos-essentials-dev & MOOSUtilityLib, header files \\ 
moos-instruments & iActuation, iLBL, iCompass, ... \\ 
moos-navigation-and-control & pHelm, pNav, iRemote \\ 
moos-navigation-and-control-dev & MOOSNavLib, MOOSTaskLib, header files \\ 
moos-thirdparty-dev & FLTKVW (library and headers), newmat headers \\ 
moos-tools & uPlayBack, uMS, uMVS \\ 
moos-tools-matlab & iMatlab \\
\hline
\end{tabular}
\end{center}

By Paul Newman's request, the packages were defined by the top-level directories
within the MOOS source code tree.  The only exception was 
\verb|moos-tools-matlab|, which was separated out because of its unique
build-time requirement (an installed copy of Matlab).

Documentation (\verb|-doc|) packages weren't included because as of this 
writing, the \verb|Doc| subdirectory within the MOOS source code tree is not
under CMake control.  Once each document in the \verb|Docs| directory has its own
\verb|CMakeLists.txt| file, this omission can be easily addressed.

\subsection{Locations of installed files}
Because of my ignorance about MOOS's policy regarding which changes in 
release version numbers indicate or counter-indicate a change in MOOS's
libraries' APIs, I took a very conservative approach to the location in
which these Debian packages will install their files.

Rather than placing the files directly into \verb|/usr/lib|, \verb|/usr/include|, 
and \verb|/usr/bin|, the Debian packages will place their files into 
\verb|/usr/(version-number)/lib|, \verb|/usr/(version-number)/include|, 
and \verb|/usr/(version-number)/bin|.
This reduces the possibility of problems stemming from unplanned API breakages
and of confusion regarding which version(s) of the software are being employed
at a given time.

Someone on the MOOS project who is also familiar with Debian packaging 
standards should probably review this decision as soon as is convenient.

\section{Modifications to MOOS' Build System}
\subsection{Shell scripts now, CPack later}
MOOS uses the CMake build system (version 2.4 is available as of this writing)
to manage the build process.  CPack is a complementary piece of software designed
to produce software that will install onto target systems the software you're
developing.  

Producing Debian packages (in addition to Windows installers, .tar.gz files, 
etc.) is the exact kind of job CPack was designed for.  Unfortunately CPack
doesn't produce Debian files at the time of this writing.  However, CPack 2.6
is expected to be released in late 2007, and assuming it's reasonably bug
free, it should be the best tool for producing MOOS's Debian packages once
it's widely available.

Several CMake modules have been written to provide Debian package creation
abilities to projects that use CMake.  However when I reviewed these I
judged none of them to have the quality and/or features required for our project.

Therefore the first pass at producing Debian packages for the MOOS software is
accomplished using a combination of enhancements to the project's existing
CMake files, some (text) template files, and a Bourne shell script.  The details
of this approach are explained in the rest of this section.

\subsection{Modifications to MOOS's CMakeLists.txt files}
The only significant change I made to MOOS's \verb|CMakeLists.txt| files was to
add a number of \verb|INSTALL(...)| directives to the individual programs' and
libraries' \verb|CMakeLists.txt| files. (That is, to the files at the leaves of the
source code directory tree.)

CMake's \verb|INSTALL| command supports the notion of an \textit{install 
component}.  An install component is a named subset of the files that would
be installed if you ran the command \verb|make install|.  (The particular
installation component that will be installed when you run \verb|make install|
is specified by the value of the \verb|CMAKE_INSTALL_COMPONENT| CMake variable
at the time you configure your Makefiles using \verb|cmake|.)

I used named components to specify the different subsets of programs, 
libraries, header files, etc. that should appear in each of MOOS's different
Debian packages.

\subsubsection{Why use CMake named installation components?}
I used named installation components because of the way \verb|dpkg-deb| works.
To use \verb|dpkg-deb|, you first populate a directory tree to look the way you
want your files to appear when someone installs your package.  For example, if 
you're using /tmp/foo/ as a working directory while you do your packaging, you
might create a set of files like this:
\begin{verbatim}
   /tmp/foo/usr/bin/pAntler
   /tmp/foo/usr/lib/libMOOS.a
\end{verbatim} 

You would then put a few other special files into the \verb|/tmp/foo/| directory 
as required by the Debian package system, and then run the \verb|dpkg-deb| program
with the directory \verb|/tmp/foo| as one of its command-line arguments.

It turns out that CMake's named installation components is very helpful in populating
the directory tree whose root is \verb|/tmp/foo/|.  You just build MOOS, and then
use a variant of \verb|make install| to specify the installation component whose
files you want to install (e.g., to appear in the Debian package you're producing)
and tell the installer that you want the installation to use \verb|/tmp/foo| as a
base directory into which the installation occurs.

\subsubsection{CMake installs: Specifying component and location}

When you run \verb|cmake| to set up a build tree / build system for your project, 
you can specify numerous details that become fixed within your Makefiles.  These
details (unlike some others) cannot be changed merely by setting certain environment
variables prior to invoking \verb|make|.  You must re-run \verb|cmake| in order to
change these details.  One such detail is the specification of which installation
component will be installed when you run \verb|make install|.

This is a problem for us, because as we build multiple different Debian packages
we need to adjust which installation component will be installed when our packaging
shell script invokes \verb|make install|.  There may be safe and acceptable ways to
re-run \verb|ccmake| each time our shell script moves on to produce a different 
Debian package, but none was obvious as I wrote the software.  This led to the following
solution:

CMake-generated \verb|make install| make targets work in a perhaps surprising way.
When you run \verb|make install|, the Makefile invokes \verb|cmake| and instructs it
to execute a CMake script file (\verb|cmake_install.cmake|) that was produced when 
you last used \verb|cmake| to configure your Makefiles.  

You can install the MOOS software by manually telling \verb|cmake| to execute the \verb|cmake_install.cmake| script, rather than doing so by running \verb|make install|.
In doing so you have the opportunity to change which installation component will be 
installed.  And this is exactly what our shell script 
\begin{verbatim}
    BuildScripts/DebianPackaging/make-debs
\end{verbatim} 
does.  

This approach works, but it's undesirable in the long term for several reasons.  
First, it's complicated and relies on some underdocumented features of CMake.
Second, according to the CMake team this approach is is unsupported, which is
another reason to switch to using CPack 2.6 once it's widely available.

\subsection{The make-debs shell script}
I've created a new Bournse shell script within the project's source tree:
\begin{verbatim}
    BuildScripts/DebianPackaging/make-debs
\end{verbatim} 

This script takes a lot of parameters, and it requires that 
you've already successfully built MOOS in some directory.  But
given those pre-requisites, this script should do a pretty good
job of building all of MOOS's Debian packages and putting them
into a directory you specify.

This script is manually invoked, rather than being executed to
satisfy a Make target.  For various reasons, attempts
to fold this functionality into MOOS's CMake files to make the
CMake files overly complex.  I judged that keeping MOOS's build 
system simple and comprehensible was more important than finding
some way to have the command \verb|make debs| work.

\subsection{File templates}
The directory tree that contains the files to be packaged also
must contain a few additional files that wouldn't normally be
considered part of the proper MOOS project.

They are:
\begin{description}
 \item[DEBIAN/control]  This specifies some details about the package that are
	pretty stagnant, such as a human-readable description of what the 
	package provides, as well as some details that change frequently such
	as the version number of the software being packaged.

	This is the one file in the package that will not
	by copied onto the file system of the target computer when the package
	is installed.

	The format of this file is described in Section 5 of the Debian Policy
	Manual:

	\begin{verbatim}
	 http://www.debian.org/doc/debian-policy/index.html
	\end{verbatim} 
 \item[/usr/share/doc/(package-name)/copyright]
	This is a copyright statement regarding the files in the package.
	As of this writing, all of MOOS's Debian packages use the same version
	of this file.

 \item[/usr/share/doc/(package-name)/changelog.Debian.gz]
	This is a change log describing the changes that have occurred 
	from one revision of the software to the next.  This file has a
	very particular format, described Section 4.4 of the Debian
	Policy Manual. 

These versions of these files that appear in the Debian packages are all 
generated from files in the MOOS source tree's
\begin{verbatim}
BuildScripts/DebianPackaging/FileTemplates\end{verbatim} 
directory.

Each \verb|CONTROL-...| file, one per Debian package, is copied and
customized by \verb|build-debs| to become a Debian package's 
\verb|DEBIAN/control| file.  \verb|build-debs| generally modifies
each of these files, as it's being copied, to inject into the file
the desired version number of the package being built.  This lets
\verb|build-debs| save a human the work of modifying these files 
as part of the Debian packaging exercise.

Currently there's only one copyright file:
\begin{verbatim}
COPYRIGHT-mit-and-oxford-gpl
\end{verbatim} 
I wrote this file based on the copyright statement I found at the top of one
of the MOOS source code files.  This file's exact text appears in the appropriate
\verb|/usr/share/doc/...| subdirectory of systems that install the MOOS 
Debian packages.

Each \verb|CHANGELOG-...| file, one per Debian package, is copied and
compressed by \verb|build-debs| to become a Debian package's 
\verb|changelog.Debian.gz| file.  The \verb|CHANGELOG...| files 
are meant to be maintained
by a human each time the MOOS project is about to release a new version of its 
software.  If these aren't maintained, users of the Debian packages will not be
able to use the existing package management tools to review the package change
history and to decide whether or not they want to install a new version of the
MOOS packages.

 \end{description}

\section{Creating Debian Packages}
\subsection{Pre-requisites}
Here's what you need on your computer to be able to produce MOOS' Debian 
packages:
\begin{description}
 \item[Debian-based operating system]
	The general wisdom is that to build Debian packages, you're best off
	doing so on a Debian-based (Debian, Ubuntu, etc.) operating system.
	Doing so ensures you have ready access to the
	tools needed to produce Debian package files (\verb|dpkg-deb|, 
	\verb|lintian|, etc.)
 \item[Ability to build MOOS]
	Building the MOOS software is an integral part of producing MOOS's
	Debian packages.  Refer to MOOS documentation regarding what this
	entails.
 \item[Software used for creating packages]
	You need the following programs installed on your system:

	\verb|dpkg-deb| This is used to create the actual Debian package files.

	\verb|lintian| This is used to confirm that a package you just build
	conforms to Debian's rather strict packaging rules.
 \item[Matlab (optional)]
	The MOOS project includes a Matlab interface: a shared library
	named \verb|iMatlab|.  To build this shared library you need a copy of
	Matlab installed on your system (I'm not sure which version(s) of 
	Matlab suffice).  

	If you don't meet this requirement you can still
	build most of MOOS's Debian packages; just not the \verb|moos-tools-matlab|
	package, which provides \verb|iMatlab|.  \verb|iMatlab| has been placed
	ni its very own Debian package so that if you don't own Matlab you're 
	still able to produce Debian packages for every other piece of MOOS 
	software.
 \end{description}

\subsection{Steps to Create Packages}
\begin{enumerate}
 \item Configure and build MOOS:

	Build MOOS.  Ensure that the build system is able to build those 
	subprojects which depend on the \verb|FLTK| library.  The current
	version of the MOOS system will simply skip those projects if it
	can't find the needed \verb|FLTK| libraries and headers, as well as the
	related \verb|fluid| program (which comes in a separate Ubuntu package,
	by the way).

	You should be sure to enable the building of every part of MOOS, with
	the optional exception of the \verb|BUILD_MATLAB_TOOLS| piece.  Otherwise
	some of the Debian packages may lack expected files and/or not build
	at all.
 \item Create (or just use) to working directories:

	One directory will be where \verb|make-debs| does its scratch-work.  The
	other directory will be where \verb|make-debs| places the completed
	\verb|.deb| files.

 \item \verb|cd| to the \verb|BuildScripts/DebianPackaging/| subdirectory within
	your MOOS source code tree.
 \item Run \verb|make-debs|

	Run the \verb|make-debs| script.  If you don't know what it's command-line
	arguments should be, run it without any arguments and help will be printed
	to the console.

	Note that if you didn't build the \verb|iMatlab| library (that is, when
	you configured the MOOS build system you set \verb|BUILD_MATLAB_TOOLS| 
	to \verb|OFF|), then you \textit{must} use the \verb|--no-matlab| option
	when invoking \verb|make-debs|.  Otherwise \verb|make-debs| will fail
	when it tries to build the \verb|moos-tools-matlab| package.
\end{enumerate}


\section{Suggested Near-term Future Work}
\begin{itemize}
 \item Review the package descriptions and dependencies as stated in the
	\verb|CONTROL-...| files.

 \item Review the \verb|COPYRIGHT-mit-and-oxford-gpl| to confirm that the
	copyright statement communicates what it's supposed to communicate.

 \item Review the decision to include the specific MOOS version number in
	the installation path for the packaged files.

 \item MOOS's \verb|Docs| directory isn't currently built by the CMake
	build system.  Once it is, it would be easy to create one or
	more \verb|-doc| documentation packages to accompany the other
	MOOS Debian packages.  This would be helpful to would-be MOOS
	users, as it reduces the burden needed to find useful documentation.
 
 \item \verb|lintian| issues warnings because the packaged MOOS programs don't
	include manpages.  Creating these manpages would be useful to users,
	and would reduce the number of \verb|lintian| warnings we see.  
	Including manpages would also add an air of robustness and polish
	to MOOS that may help persuade people it's worth investing time in
	learning to use.

 \item MOOS uses a customized CMake module called MOOSFindFLTK.  This package,
	as well as the standard FindFLTK that's disbributed with CMake 2.4,
	seem to have some problems that make it unnecessarily difficult
	to build MOOS but could be addressed:

	\begin{itemize}
	 \item At least on Ubuntu 7.04, this package fails to find FLTK even
		when it's installed.  I've gotten around this by setting the
		\verb|FLTK_INCLUDE_DIR| CMake variable to have a value of
		\verb|/usr/include/|.  But when I don't do this or something
		else to fix the problem, MOOS builds with only the meekest of
		warnings that it's going to skip building those modules that
		require FLTK.

	 \item After invoking this module, MOOS's build system looks at the
		\verb|FOUND_FLTK| CMake variable to decide whether or not the
		FLTK-dependent parts of MOOS can be built.  But this is overly
		conservative.

		\verb|FOUND_FLTK| will be set to \verb|1| if, in addition to
		the libraries and headers being found, the program \verb|fluid|
		was also found.  As far as I can tell, MOOS doesn't use 
		\verb|fluid|.  But \verb|fluid| appears in its own Debian
		package; if a user hasn't installed the (unnecessary)
		\verb|fluid| Debian package, MOOS will unnecessarly skip
		building its FLTK-dependent parts.

		This is a problem because to the typical programmer trying 
		to build MOOS, it's not at all clear that the solution to the
		problem of FLTK-dependent parts not buiding is to install a
		program named \verb|fluid|.
	\end{itemize}

\end{itemize}


\section{Suggested Medium-term Future Work}
\begin{itemize}

 \item Switch this whole mess over to using CPack 2.6 when it comes out.

 \item Either create an \verb|APT| repository to host MOOS's packages, or get them
	added to Debian's \verb|universe| repository group.  This would make it
	\textit{trivially} easy to introduce MOOS to Debian / Ubuntu / etc. users.

\end{itemize}


\end{document}
